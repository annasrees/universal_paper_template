\begin{figure*}
  \centering
  \includegraphics[width=0.8\textwidth]{cartoon_highlight.pdf}
  \caption{
    Partioning of Charles Dickens's ``Tale of Two Cities''
    into five different kinds of elements: 
    clauses (red),
    pure random partitioning phrases ($\partitionprob=\frac{1}{2}$, orange),
    words (yellow),
    pure random partitioning graphemes ($\partitionprob=\frac{1}{2}$,
    green),
    and
    letters (blue).
    \textbf{A.}
    Partition examples for the start of the work.
    The specific phrases and graphemes shown are for one realization
    of pure random partitioning.
    \textbf{B.}
    Zipf distributions for the five kinds of partitions
    along with estimates of exponents.
    See supplementary material for measurement details.
    \todo{add connecting line for letters only as a guide}
    \todo{remove estimate for letters? not signficant?}
    \todo{2 decimal places with errors.}
    \todo{Change 'slope' to $\simonalpha$.}
   }
  \label{fig:tpz.twocities}
\end{figure*}



\todo{Insert note that clauses are the most independent and follow something like a Simon's model}


Continuing with the statistical mechanical analogy,
we see that because random partitioning can be viewed as an infinite 
temperature limit ($\beta=0$, $T=\infty$), we may be reassured that the resulting Zipf
distribution is robust for a wide range of adjacent partitions (i.e., $\beta$ near 0).
The traditional zero temperature word-centric partition 
($\beta \rightarrow \infty$, $T \rightarrow 0^{+}$) 
both performs poorly and lacks stability.
In allowing the non-physical but abstractly allowable adjustment to
$\beta \rightarrow -\infty$ ($T \rightarrow 0^{-}$),
we would have clauses partitioned only as themselves,
and the resulting distribution of so many disparate, unique entities
would depart even further from a pure scaling form.\\
\todo{Fix the above up given new figures.}
\todo{Can we say this?}\\
We note that the appearance of a power law is not guaranteed.
As a simple example, if we apply random partitioning to a repeating sequence
of the same word, say `ook', then the resulting rank distribution
will be exponential.


\begin{figure}
  \centering
  \includegraphics[width=\columnwidth]{twoCitiesVertCartoon.pdf}
  \caption{
    \todo{Add caption.}
    \todo{Measure exponents.}
    This is one random 
  }
  \label{fig:tpz.twocities}
\end{figure}


%% Here, we present a fast, efficient, and scalable method for
%% partitioning arbitrary, large-scale texts into meaningful words and phrases.

%% n-grams
%% requiring our collective analysis 


\todo{Reconciling with text-mixing: Decay of phrase introduction is
  slower when phrases are involved.  Word order matters.}

\todo{Do following for Fig. 1}\\
\todo{Please convert to $\log_{10} r$, etc.} \\
\todo{Remove bold from headings}\\
\todo{increase font size for axis labels}\\
\todo{Underlay main figure with normal Zipf distribution?
  That is, take the lower subplots out.
  Could have the normal one offset lower.
  Main thing is to make the comparison as clear as possible.}


For each corpus we observe a scaling
break of reduced severity following the regime over which Zipf's
law (black line main plot, dashed red in the lower left) is
present. 


This reduction in the severity of the lower tail is
consistent with text mixing theory, as differences of word sets
between the texts of a corpus are compunded with word order, which
creates a greater diversity of symbols---hence a decreased rate of
symbol-introduction decay---between texts. 

Also, we observe that
despite having a greater number of unique words, the Twitter
dataset has fewer partitioned elements than either the New York
Times or 
This is possibly due to the
character limit on the tweet lengths, and the fact that alphabetic
sequences are used in creating URLs, which, despite having little
semantic
meaning are picked up by our parser as word forms.


\begin{figure}[tbp!]
  \centering
  \includegraphics[width=.47\textwidth]{Grandma.pdf}
  \caption{
    The possible random partitions of the string `let's eat
    Grandma' ($\sigma$).
    Each partition is equally likely giving rise
    to expected weights for the phrases they contain.
    The six phrases along with their weights are
    `let's' (1/2),
    `eat' (1/4),
    `Grandma' (1/2),
    `let's eat' (1/4),
    `eat Grandma' (1/4),
    and 
    `let's eat Grandma' (1/4).
    These weights sum to that the number of words 
    in the string, $n_\sigma=3$.
    \todo{Add weights for random partitions}
    \todo{Choose a more friendly phrase?}
    \todo{Use / or $\mid$ instead of comma}
  }
  \label{fig:textpart.grandma}
\end{figure}


We show a simple example of random partitioning for the
punctuation-sensitive phrase `let's eat Grandma'.  in
Fig.~\ref{fig:textpart.grandma}.



Random Partitioning gives us a useful way to measure
the `strength' of a given phrase $\tau$ with respect to its local contexts---what we will
call $\tau$'s \textit{local context conditional entropy} $\lcce_\tau$---in the following way.
First, we define the phrase neighborhood $\mathcal{N}_\tau$
`generated' by a phrase $\tau$ as the collection of
all length $n_{\tau}+1$ phrases found in the text $T$
that can be formed from $\tau$ by an insertion of a single word.
We in turn define the neighborhoods to which a phrase 
owes membership as its \textit{internal contexts}.
Our example phrase `let's eat Grandma' 
is thus a member of three internal
contexts (if they exist in $T$): 
$\mathcal{N}_{\textnormal{let's eat}}$, 
$\mathcal{N}_{\textnormal{let's Grandma}}$, 
and $\mathcal{N}_{\textnormal{eat Grandma}}$.
Denoting a context as $C$ and the set of all
internal contexts of $\tau$ as $\mathcal{C}_\tau$, we define 
$\lcce_\tau = \max_{C \in \mathcal{C}_\tau} - \log_{2} P(\tau \mid C)$.
Thus, $\lcce_\tau$ is derived from 
most unlikely and most information rich
inner context of $\tau$.

\todo{Include somewhere?}: Though there is no reason to assume that a distribution of text
separated $\&$ rank-order phrase counts abides the Zipfian hypothesis,
it is known that a pure Zipfian word distribution can only allow for
non-Zipfian $n$-gram distributions \cite{egghe????a}.

\todo{Include somewhere?:
A great deal of research has been done in the
way of MWE extraction, where most successful developments often rely
upon parallel corpora.  
In this investigation we provide an extraction
technique that relies upon the text being interpreted alone.}


\section{Discussion}
\label{sec:textpart.discussion}

A perhaps understated feature of the Serial Partitioning algorithm is
the means by which likelihood data might be accessed.
The method relies upon local likelihoods defined by the internal
context model. 
These likelihoods are determined according to the tree structure
proposed in~\cite{???}.
Interestingly, one could just as well store these likelihoods on the
aforementioned tree.
Accessing this data then becomes computationally efficient (searches
are log-time on data trees), which allows for biologically feasiblity. 

As mentioned in~\cite{williams2013a} language occurs on a multitude of
scales. 
This paper focuses on the phrase scale, for its position as a highly
active regime.
Serial Partitioning provides a novel technique for the detection of
meaningful language units.  
In the N.L.P. community these techniques could easily be implemented
to build better translation services with no need for pre-translated
parallel corpora.  
From any serially encoded text, our method extracts likely lexical
content - providing a means to direct the (context-dependent)
identification and translation of idioms in the most advanced
M.T. techniques.

In the phrase distributions section above we describe the distribution
of expected counts.
For a text, this distribtuion is defined by a random and independent
partitioning of text strings into contiguous sequences of words.
Here we perform precisely this operation, heretofore referred to as
random partitioning.

Our Serial Partitioning algorithm is in large effective due to a
strong conditioning present in our context model.
This conditioning on the words is precisely a realization of the
non-independence of word appearances as one reads a text.
Hence, the words of a text do no appear in accord to the original
Simon model~\cite{simon1955a}, i.e. independently.

As a premise for this work, we supposed that meaningfully partitioned
phrases are directly interpretable for meaning.
These phrases

This can be explained by the independance of partitioned phrases from
any other previously partitoned phrases. 
So phrases of a meaningfully partitioned text appear more
independently than the words.

 
This then is our heuristic justification for why informed text
partitions adehere to Zipf's law more so than words.
What is more, we expect that the phrases of a random partition will be
even more independent (they are without reference to possible
meaning-dependencies), and hence be even more in accord with the
Zipf-Simon relationship.

We present a random partition performed on the NY times text (Figure
2).
Note that this distribution posseses a small number of units scaling
more shallowly than the $-1$ exponent, and then rigidly follows a
scaling quite close to $-1$.
these two regions are indeed in direct accord to the analysis of the
original Simon model.
To test the possibility that a random partition is a realization of
the Simon model, we perform Kolmogorov-Smirnov \todo{check spelling
  and perform test} tests on randomly partitioned texts against
distributions of the same size generated as per the Simon model (Table
2).
\todo{make table of K-S test results for various innovation rates}



%%%%%%%%%%%%%%%%%%%%%%%%%%%%%%%%%%%%%%%%%%%%%%%%%%%%%%%%%%%%%%%%%%%%%%%%%%%%%%%%%%%%%%%%%%

\todo{Cite Sole papers}

\todo{Explain partitioning in one or two paragraphs}

\todo{Describe empirical results}

\todo{Describe Model}

\todo{Describe Analysis}

\todo{Give results of model}

\todo{Discussion}



\todo{Include music lyrics again?}


\todo{Show that Random Partitioning does a good job for Zipf?}

In Fig.~\ref{fig:textpart.mobydick_randompart}, we compared
the standard Zipf distribution for Moby Dick with the 
phrase weight distribution we obtain from randomly partitioning.\\
\todo{Expand on these results.}\\
\todo{Add NYT or Wiki?}


\begin{figure*}[tbp!]
  \centering
  \includegraphics[width=\textwidth]{2013-10-generatedLanguage.pdf}
  \caption{
    A. Generated language word distribution. 
    B. Generated language phrase distribution. 
    Both word and phrase distribution upper ($\gamma_{1}$) and lower ($\gamma_{2}$) 
    scaling exponents reported in graph. 
    C. The percentage of usage of words by order (1,\ldots,10 represented cold to hot), 
    with both rank and inception as predictor variable. 
    D. Grey bars indicate the percent of all units counted that each order possesses, 
    and red bars indicate the percent of all unique units each order possesses. 
    E. A log-log heat map of word inception vs word rank. 
    F. Shannon phrase (red) and word (black) distribution entropies over of time. 
    G. Phrase (red) and word (black) distribution nullity over time.
  }
  \label{fig:textpart.generatedlanguage}
\end{figure*}


\todo{Lead into why we must go to serial partitioning}

\textbf{Local Serial Partitioning}---Random Partitioning gives us a background distribution
of phrases to work with, but one that clearly does not account for how a
reader might parse text
(necessarily working with limited cognitive resources~\cite{levy????a}),
but rather provides us with only an ensemble of possible partitions.
To capture the act of interpretation, we introduce a local serial
partitioning method 
which `reads' through strings to find phrases which
maximize entropy local context conditional entropy.

\begin{figure}[tbp!]
  \centering
  \includegraphics[width=.47\textwidth]{Grandma.pdf}
  \caption{
    The possible random partitions of the statement `let's eat Grandma'.
  }
  \label{fig:textpart.grandma}
\end{figure}


Below, we step through an example of Serial Partitioning for the string `let's eat Grandma'.
Starting at the left, we set our first partition to be the null phrase
$\varphi$ 
which has local context conditional entropy $\lcce_\varphi =  0$.
The first word is concatenated onto our partition, producing the phrase `let's',
whose positive entropy keeps the process going.
The word `eat' is then concatenated to form `let's eat'. 
Since the entropy increases, we keep the phrase and continue.
Upon adding the word `Grandma', the entropy decreases, 
and we fix the phrase `let's eat' as the first partition.
The process begins again with the null phrase $\varphi$.
The next phrase continues with the word `Grandma' and the partitioning
is complete (and ethically sound).
\begin{enumerate}
\item let's eat Grandma /  $\lcce_\varphi=0$
\item $\underbracket[0.5pt]{\textnormal{\textbf{let's}}}$ eat Grandma /  $\lcce_\text{let's}>\lcce_\varphi=0$
\item $\underbracket[0.5pt]{\textnormal{\textbf{let's eat}}}$ Grandma /  $\lcce_\text{let's eat}>\lcce_\text{let's}$
\item $\underbracket[0.5pt]{\textnormal{\textbf{let's eat Grandma}}}$ / $\lcce_\text{let's eat Grandma}$$<$$\lcce_\text{let's eat}$
\item $\underbracket[0.5pt]{\textnormal{let's eat}} \,|\,$ Grandma /  $\lcce_\varphi=0$
\item $\underbracket[0.5pt]{\textnormal{let's eat}} \,|\,$ $\underbracket[0.5pt]{\textnormal{\textbf{Grandma}}}$ / $\lcce_\text{Grandma}>\lcce_\varphi= 0$
\end{enumerate}
By moving through the full text $T$, treating string by string 
with Serial Partitioning, we extract a unique partition of $T$ into words and phrases.

\begin{figure*}[tbp!]
  \centering
  \includegraphics[width=\textwidth]{dick_random_english_counts.pdf}
  \caption{
    \todo{Finish caption.}\\
    \textbf{A.} Word frequency as a function of rank for Moby Dick.
    \todo{Describe inset in A.}
    \textbf{B.} The weight distribution for Randomly Partitioned phrases.
    }
  \label{fig:textpart.mobydick_randompart}
\end{figure*}

\begin{figure*}[tbp!]
  \centering
  \includegraphics[width=\textwidth]{2013-07-times_counts.pdf}
  \caption{
    \textbf{A.} The distribution of words from the New York Times corpus.
    \todo{Describe inset in A.}
    \textbf{B.} Serially partitioned phrases from the New York Times corpus. 
    An ideal Zipf distribution will appear as a straight line of slope $-1$ (red dotted line for reference). 
    The inset bar plot shows the percent of all unique symbols
    partitioned at each phrase length (pink), 
    and the percent of the total number of units partitioned at each
    order (grey)
    \todo{more clarify in preceding.}
    The phrases presented are ranks $10^n$ ($n=0,...,6$) in both the word and phrase separated distributions. 
    Even at high ranks, phrase units are interpretable (often idiomatic) regular language components, 
    while high ranking words like `cabbage', `mysteriousness' and
    `fish-slicing' are difficult to glean meaning from outside of
    context.\newline
    \todo{describe double fit and give errors}\newline
    \todo{Change order to 'phrase length' in plot}\newline
    \todo{Move slopes to be near actual fits}\newline
    \todo{Move example phrases so they are attached to points on plot}\newline
    \todo{Add examples of 3, 4, 5, 6 word phrases?}
    }
  \label{fig:textpart.nytimes}
\end{figure*}


We now employ this Serial Partitioning scheme to examine
two disparate, large-scale corpora: 
the English version of Wikipedia and the New York Times (see Methods).
We show how the standard word frequency distribution compares with
the serially partitioned  phrase distribution for the New York Times in
Fig.~\ref{fig:textpart.nytimes}
(see Fig.~\ref{fig:textpart.wikipedia} for the Wikipedia plots).
The characteristic scaling break (or `knee') of the Zipf  word distribution
(Fig.~\ref{fig:textpart.nytimes}A)
is in strong evidence, occurring between word ranks of $10^3$ and $10^4$.
For low rank words, frequency $f_r$ scales as $r^{-0.98 \pm ??}$, in
accordance with Zipf's original observations~\cite{zipf1949a} and many that have followed.
Above the break, the slope becomes much steeper with 
$f_r \sim r^{-1.85 \pm ??}$.
\todo{Describe the inset in A}

But for both corpora, this break in scaling largely evaporates
for the distribution for phrases.
For the New York Times, the scaling for infrequent phrases 
follows  $r^{-1.19 \pm}$, weakly different from $r^{-1}$ (Fig.~\ref{fig:textpart.nytimes}B).
Points are colored according to phrase length, revealing a gradual,
mixed transition through phrases of increasing length as rank increases.


The reduction of a break in scaling comes about 
becomes words that appear frequently and are very likely to appear in sequence
will agglomerate into higher order units (see `new york', `york city', and
`new york city' in Fig.~\ref{fig:textpart.nytimes}B \todo{Bring these back?}). 
This process not only diminishes the frequency of the individual words
(suppressing the scaling break), but introduces new, lower-frequency
atoms (phrases with two or more words),
which fill out the tail.

\todo{Explain insets}

Apart from the phrase distribution's close adherence to an idealized Zipf's
relationship, we also find that the most frequently occurring phrases are
interpretable, quite representative of the source corpora, and often
idiomatic (see Tabs.~\ref{tab:textpart.nytimesphrases} and~\ref{tab:textpart.wikipediaphrases}).
Given the relatively simple nature of Serial Paritioning,
the dramatic movement toward a pure Zipf distribution along with the extraction
of reasonable phrases suggests that improved or alternate methods of
partitioning should produce similar results.

\todo{Add more description}


Plotting the resulting phrase distributions for phrases up to length
10, we see that the Zipfian
hypothesis is supported remarkably well.



\section{Stochastic Model}
\label{sec:textpart.model}



In~\cite{araujo2013}, the argument is made that a `mature' natural
langauge is of a Ziphan statistical structure.
Their argument hinges on the property that the Zipfian (harmonic)
distribution has an entropy sum that diverges (in the limit).
Consequently, a Zipfian distributuion may have an arbitrarily large
average entropy as more units are incorporated.
The perspective then is that there is little rank-mobility; the
language is matured or saturated.

However as we and others~\cite{ha2009a,gerlach2013a} have observed, large
enough samples of text produce a break in scaling in their rank-frequency
distributions.
Though some hold the perspective that those words beyond the scaling break are
`junk'~\cite{???}, we view this as case in point of a
non-Zipfian word distribution.
Assuming then that the word distributions are in some way non-Zipfian,
we develop a model based on Simon's~\cite{simon1955a}.
In this model we seek to determine what language development processes
can lead to the formation of a scaling break.

Integral to this development are two interacting distributions: a
`phrase distribution' of constructed units, and a `word
distribution' that is the atomization of the former.
With each time step there are three probabilities, $\alpha$,
$\beta_1$, and $\beta_2$ that govern the construction process. With
each step one will:
\begin{itemize}
\item[$\cdot$] 
  choose a word with probability $\beta_1$;
\item[$\cdot$] 
  choose no word with probability $1-\beta_1$;
\item[$\cdot$] 
  choose a phrase with probability $\beta_2$;
\item[$\cdot$] 
  choose no phrase with probability $1-\beta_2$;
\end{itemize}
where if a word is chose, it will be
\begin{itemize}
\item[$\cdot$] 
  a new word with probability $\alpha$;
\item[$\cdot$] 
  an old word with probability $1-\alpha$.
\end{itemize}

In general any phrase chosen will be an old one, and old words and
phrase are chosen in proportion to the number of times they have
previously appeared. If no word and no phrase are chosen, then the
process is null, rendering no phrase. If both word and phrase are
chosen, then the two will be composed to form the newly constructed
phrase that is counted into the phase distribution, and whose
constituent words are counted into the word distribution.

Perhaps the most straightforward manner to compose phrase and word
then would be to concatenate the word onto the phrase, but, as it
changes our analysis of the resulting word distribution in no way, we
compose the word and phrase together in a more nuanced manner.
In particular, if our phrase has length $n$, we keep track of the
insertion frequncy of usage of each of the $n+1$ possible insertion
points (every phrase insertion history is initialized to a nuiform
state upon its inception). Then, chooseing an insertion point in
proportion to its frequency of usage, the word is inserted
accordingly.


\section{Model Analysis}
\label{sec:textpart.modelanalysis}

To describe the the growth of the word distribution in the stochastic
model presented in the previous section we define $\mathcal{N}_k$ to be the
number of $k$-count words in the word distribution, $n_w$ to be the
total number of words in the word distribution, $n_p$ to be the number
of phrases in the phrase distribution, and $M_k(i)$ to be the number
of (non-unique) phrases in the phrase distribution with $i$ words of
word-count $k$. Then, measuring time by $t$, the number of non-null
construction steps, the growth in $\mathcal{N}_k$ per non-null construction
event is given by
\begin{equation}
\frac{\rm{d}\mathcal{N}_k}{\rm{d}t}
=
\beta_1\frac{\rm{d}\mathcal{N}_k}{\rm{d}n_w}
+
\frac{\beta_2}{n_p}
\left[
  \sum_{i=1}^\infty iM_{k-1}(i)-\sum_{i=1}^\infty iM_{k}(i)
\right],
\end{equation}
where as in~\cite{simon1955a,krapivsky2001a} we know
$$\frac{\rm{d}\mathcal{N}_k}{\rm{d}n_w}=\frac{1}{n_w}\left[(k-1)\mathcal{N}_{k-1}-k\mathcal{N}_k\right](1-\alpha)+\alpha\delta_{k,1}$$
and similarly assume that $n_wp_k=\mathcal{N}_k$. On the left hand side we
immediately find the differential,
$\text{d}n_{w}/\text{d}t$.  
So, since the $M_k(i)$ have the nice property:
$$
\sum_{i=1}^{\infty}
iM_{k}(i) 
=
k\mathcal{N}_k,
$$
the right hand side becomes,
$$
\left[
\beta_1(1-\alpha)
+
\beta_2
\frac{n_w}{n_p}
\right]
\left[
(k-1)p_{k-1}-kp_k
\right]
+\alpha\beta_1\delta_{k,1}.
$$
On average a constructed phrase will have been the result of $\beta_1$
word-drawn words and $\beta_2(n_w/n_p-\beta_1)$ phrase-drawn words. As
such we expect that the approximation
$$\frac{\rm{d} n_w}{\rm{d}t}\approx\beta_1+\beta_2\left(\frac{n_w}{n_p}-\beta_1\right)$$
will hold for large $t$. Putting this all together we find the recursion equation
$$p_k=p_{k-1}\frac{(k-1)}{k+1/\eta},$$
where
$$\eta=\frac{\beta_2\frac{n_w}{n_p}+\beta_1(1-\alpha)}{\beta_2\frac{n_w}{n_p}+\beta_1(1-\beta_2)}.$$

As usual the recursion equation is solved in terms of gamma functions,
where the resulting fit applies to the low-$k$ words. In particular,
if $t_b$ is the time at which the $b^\text{th}$ new word
appears, those words in teh vicinity of rank $b$ will scale with
exponent $-\eta_{t_b}$ in the rank-frequency distribution. The
proportion $n_w/n_p$ does grow (though slowly) over time, and the
result there is tail, initial steep, but gradually increasing.

While foregoing an analysis of the phrase distribution, we do take
time to consider what value $\eta$ may take. Note that if $\beta_2=0$
the model collapses back to that of Simon, with rank-frequency
exponent $\eta=1-\alpha$. In general, we see that if $\beta_2>\alpha$
one ahs $\eta>1$, and that $\eta$ indeed rises with $\beta_2$. If
$\beta_2$ is quite nearly $1$, we obtain
$$\eta\approx1+\beta_1(1-\alpha)\frac{n_w}{n_p},$$
which has the upper limit $1+n_p/n_w$ as $\beta_1\rightarrow 1$ and
$\alpha\rightarrow 0$.

\section{Simulation Results}
\label{sec:textpart.simresults}

We find that indeed the analysis of our model aligns with those runs
observe, where the fit is indeed better as $t$ becomes large. In fact
the fit converges faster for larger values of $\beta_2$ (the phrase
construction process is more dominant then). A peculiar aspect of the
model as presented is that the words with the highest count will
frequently reside in a family of very repetitive phrases. Aside from a
small class of lyrics and chat phrases in our partitioning experiments
(e.g. `ha ha ha...', `la la la...', etc.), most realistic phrases
do not involve word reptition (and certinly no the mose frequent
ones). As such a more nuanced model may have some form of repetition
inhibition. As seen in Fig. 4, such a low-reptition accommodation does
not cause the model to deviate from the analysis, while producing a
more reasonable collection of phrases.

In th previous section our analysis shows that our model is capable of
producing rank-frequency exponents steeper than $-1$. Further, since
$\beta_2=0$ produces the Simon model we may create a knee (see Fig. 4)
in the word distribution via the stepwise procedure: initiate the
model with $\beta_2=0$ to produce a Zipf-like scaling of $f\propto
r^{1-\alpha}$, and after the $b^\text{th}$ word appears begin
the phrasing process by setting $\beta_2>0$.

Supposing the growth of a language follows the course outlined above, it 
initiates with words forming and recurring independently, i.e. the
Simon model. For this period the words bear no relation to one another, 
i.e. are free of interword context. At some point a language participant 
changes the game, lifting a sequence of words from their literal meaning, 
now as an entity of its own. In the war of communication the process
snowballs with other participants borrowing meaning from existing words
to build their own colorful and abstracted interpretations. Though in
the model the point of transition is not endogenously artificed, it is
indeed conceivable in the realm of social dynamics.

%%%%%%

\todo{Sort the following out.  The main problem is that
serial partitioning does not seem to be exactly the plan
when we consider all possible internal contexts.  Also,
all decisions are made on global information, and only
in part on past experience.}

\todo{internal versus local contexts.}


%% jake:
%% http://www.longtail.com/the_long_tail/2006/09/is_zipfs_law_ju.html


In general, we consider the joint pdf for phrases and contexts
$P(\tau,C)$ with $\tau \in \mathcal{T}$ and $C \in \mathcal{C}$, their
full sets.
The conditional distribution $P(\tau \,|\, C)$
then gives the probability of a phrase $\tau$ appearing
given context $C$.
We refer to these probabilities as \textit{local likelihoods}, 
and use them as a basis for our serial partitioning method.

Let $\mathcal{T}$ be the usual collection of all phrases constructable from a
finite word set.
With our context collection $\mathcal{C}$, we must now make some
assumptions to form a valid joint pdf on $\mathcal{T}\times\mathcal{C}$.
Suppose that an interpreter experiences each internal context upon
presentation with a phrase.


For local contexts, we define the phrase neighborhood generated by a phrase
$\tau$, $\mathcal{N}_\tau$, as the collection of
all length $n_{\tau}+1$ phrases found in the text $T$
that can be formed from $\tau$ by an insertion of a single word.
We in turn define the neighborhoods to which a phrase 
owes membership as its \textit{internal contexts}.
Our example phrase `let's eat Grandma' 
is a member of three internal
contexts: $\mathcal{N}_{\textnormal{let's eat}}$, 
$\mathcal{N}_{\textnormal{let's Grandma}}$, 
and $\mathcal{N}_{\textnormal{eat Grandma}}$.

Upon presentation of the phrase
`let's eat Grandma', each of the contexts
$\mathcal{N}_{\textnormal{let's eat}}$,
$\mathcal{N}_{\textnormal{let's Grandma}}$, and
  $\mathcal{N}_{\textnormal{eat Grandma}}$ are experienced once.
So, using Random Partitioning to generate
a distribution of phrase weights for a text, we draw a
phrase from $\mathcal{T}$ in proportion to its count, and define a joint pdf
from the marginals:
$$
P(\tau)=P(\textnormal{an internal context is experienced from $\tau$})
$$
and
$$
P(C=\mathcal{N}_\tau)
=
P(\textnormal{the internal context $\mathcal{N}_\tau$ is experienced}).
$$


Note that we can represent the first marginal by weighting the counts
of each phrase by its length, but the framing does not aid in the
formation of a joint distribution, as it is without reference to
context experience.



We then take that a written body is separable, with ideal separations
accessible via some ``minimal effort'' algorithmic optimization.




Our objective is to break partition a text into meaningful units.

To the computational linguist, what we refer to here as phrases are
often called multi-word expressions (MWEs) or $n$-grams. 

The identification of MWEs in natural
language poses one of the largest obstacles to accurate machine
translation~\cite{sag????a}.

A great deal of research has been done in the
way of MWE extraction, where most successful developments often rely
upon parallel corpora.  

In this investigation we provide an extraction
technique that relies upon the text being interpreted alone.

Others have noted similar results by combining frequency distributions of n-grams,
though this approach overcounts words~\cite{ha2002a}.


A natural language speaker must consider the likelihood of a statement
or phrase's interpretation in context, that much is clear. 
But what precisely is context? 
For us it is both the previously experienced corpus and the content of
the utterance thus far. 
Previous approaches~\cite{piantadosi2011a} define the context-- for us,
\emph{local context}-- as follows. 
Given a phrase of length $n$, the context of the final word is the
sequence of $n-1$ preceding words.
In \cite{williams2013a} the bent on phrases (as opposed to just words)
allows for a more general definition. 


If words formed a layer of units of meaning,
then Simon's model would suffice for an approximate
origin of language model.


for phrase extraction to partition natural language,
we

At the core of much quantitative language analysis
is simple word counting.  

Models 

Zipf's law has 




In previous work \cite{Williams2013a} a natural language is defined
recursively, with the notion of lexicon made relative to the scale of
language being considered.
Assuming this to be the case, we ask how a text's lexicon of
meaningful phrases may be uncovered.
It is easy to ask questions about the collections of fundamental
pieces (e.g. letters in words, or words in clauses), but the trickier
business lies with `meaningful' units in clauses and words.
Since these non-trivial regimes are defined by interpretation through
similarity of past experience, we develop a model for the
identification of lexical atoms based on morphological similarity and
accumulated inferential data.




It is of some surprise then that Zipf's law is surprisingly limited
for language, applying typically to only around the first few thousand
most frequently used words whereupon a clear break in scaling
appears~\cite{bentknee,ferrericancho,ha2009a}.


\begin{figure}[tbp!]
  \centering
  \includegraphics[width=0.48\textwidth, angle=0]{mary2.pdf}
  \caption{
    X-bar theoretic decomposition of the sentence `Mary had a ball at
    her dance recital and knocked it out of the park'. Observe that
    the grammatical dissection finds both `ball' and `it' as subjects
    possessed by `Mary', acted on in conjoined verb phrases. From this
    perspective it seems that `it' is the `ball'.
  }
  \label{fig:textpart.mary}
\end{figure}


Let's observe the severity of the phrase identification problem and consider whether or not the statement
\begin{center}
\begin{minipage}{.9\linewidth}
\center
Mary had a ball at her dance recital and knocked it out of the park.
\end{minipage}
\end{center}
\noindent is so easily interpreted grammatically (Diagram 1), or
literally.  Were this the only option, we might believe that Mary had
a dance recital in a park where she somehow knocked a toy sphere
beyond the park's limits. This is troubling since the colloquial
speaker will immediately know that Mary's ball was not a toy sphere,
but instead a positive experience. The root of this concern may be
seen through a naive grammatical interpretation that implies ``it'' in
the phrase, ``knocked it out of the park'', is the ``ball'' (Diagram
1). Though ``it'' was indeed the positive experience that Mary
excelled at, equating the two easily leads a non-native speaker to the
incorrect literal interpretation, after all, a ball might certainly be
batted about. For the native speaker, knowledge of context directs
their interpretation of ``knocked it out of the park'' as a closed
structure; i.e. ``it'' is internal and fixed within an atomic phrase--
that is, it cannot be broken down any further.


\clearpage

Previous:

Let's observe the severity of the phrase identification problem and
consider whether or not the statement
\begin{center}
\begin{minipage}{.9\linewidth}
\center
Mary had a ball at her dance recital and knocked it out of the park.
\end{minipage}
\end{center}
is so easily interpreted grammatically~\ref{fig:textpart.mary}, or literally.
Were this the only option, we might believe that Mary had a dance
recital in a park where she somehow knocked a toy sphere beyond the
park's limits. This is troubling since the colloquial speaker will
immediately know that Mary's ball was not a toy sphere, but instead a
positive experience. The root of this concern may be seen through a
naive grammatical interpretation that implies `it' in the phrase,
`knocked it out of the park', is the `ball'~\ref{fig:textpart.mary}. 
Though `it'
was indeed the positive experience that Mary excelled at, equating the
two easily leads a non-native speaker to the incorrect literal
interpretation, after all, a ball might certainly be batted about. For
the native speaker, knowledge of context directs their interpretation
of `knocked it out of the park' as a closed structure; i.e. `it' is
internal and fixed within an atomic phrase-- that is, it cannot be
broken down any further.

The linguist Brame \cite{Brame01} viewed natural language as a
recursive system - one where complex relational structures build up
from smaller ones until they gel into the fundamental pieces of the
next scale. Assuming this to be the case, how do we know which
portions of an utterance are atomic and cannot be parsed further? It
is easy to ask questions about the collections of fundamental pieces
(e.g., letters in words, or words in clauses), but the thornier
territory is always in the mixing regimes (e.g., the morphemes in
words, or `meaningful' phrases in clauses). Being that these
non-trivial regimes are defined by interpretation through similarity
of past experience, we develop a model for the identification of
lexical atoms based on morphological similarity and accumulated
inferential data.

A natural language speaker must consider the likelihood of a statement
or phrase's interpretation in context, that much is clear. But what
precisely is context? For us it is both the previously experienced
corpus, and the content of the utterance thus far. Previous approaches
\cite{Piantadosi01} define the context-- for us, \emph{local
context}-- as follows. Given a phrase of length $n$, the context of
the final word is the sequence of $n-1$ preceding words. We may, by
allowing phrases (as opposed to just words) to be the context-bearing
units, find a more general definition. Instead of just the $n-1$
leading words it is well to consider any $n-1$ length subsequence of
words. Here we build a top-down approach to local context which
replaces the assumption that the role filled by a word is fixed at
its' utterance, and replaces it with a windowed flexibility. We thus
look to organize language by collections of phrases with similarly
ordered sub-structures, this being the shape of languages of recursion
grown.



\section{The tree structure of phrases and resulting metric}

Let $W$ be the collection of words in a language including the null
word, $\varnothing$, and $S$ be the collection of all finite-length
sequences of words from $W$. We will refer to elements of $S$ as
phrases. For phrases $\sigma,\tau\in S$ we define the relation
$\preceq$ by setting $\tau\preceq\sigma$ if and only if $\sigma$ may
be constructed from $\tau$ through successive insertions. For example,
if $\tau=(the\:brown\:fox)$, and
$\sigma=(the\:quick\:brown\:fox\:jumped)$, then $\tau\preceq\sigma$.

The collection of all phrases in a language is immediately an
infinite, unbounded tree - a partially ordered set of phrases
branching out from the null word. With this structure in mind, we
refer to the first neighborhood of an element $\sigma$, or $N_\sigma$,
as the set of all phrases constructible from $\sigma$ by a single
insertion, e.g., with $\sigma$ as in the previous we see that $(the\:
quick\: brown\: fox)\in N_\sigma$.

As is shown in Appendix III, the collection of phrases has a wealth of
structure endowed to it via the metric induced by $\preceq$. First one
defines the \emph{order} map $\#: S\rightarrow\mathbb{Z}$ to send
$\sigma$ to the number terms (words) it contains. Then, for any
$\sigma,\tau\in S$ define $\eta$ to be a \emph{maximal sequential
intersection} of the pair, if $\eta$ is a subsequence of both $\sigma$
and $\tau$ that has order at least as big as any other subsequence
common to the pair. Then the function $d:S^2\rightarrow\mathbb{Z}$
defined by
$$d(\sigma,\tau)=\#\sigma+\#\tau-2\#\eta$$
is a metric on $S$ (Appendix III). The main value of the metric is the
opportunity to build a notion of similarity on $S$. This is easily
accomplished by defining the normalized value
$\rho:S^2\rightarrow\mathbb{Q}$
$$\rho(\sigma,\tau)=\frac{d(\sigma,\tau)}{\#\sigma+\#\tau}.$$
The similarity function then shows us that $N_\sigma$ is for $\sigma$
the set of $\rho$-maximal similar elements (Appendix III), which we
then take as justification of the choice of the $N_\sigma$ as the
local contexts. The graph and resulting metric framework provide a
means to tease apart the regular recursive forms, but as we will see
shortly, it is these sequential perspectives in combination with
inferential data that directs the process of interpretation.

\section{Counting distributions}
In \cite{Zipf01}, George K. Zipf demonstrated that most frequently
appearing words in the English language strongly adhered to a simple
relationship. If the words were ranked ascending according to their
counts descending (i.e. the most frequently appearing word gets rank
1), then a word's rank, $R$, would be inversely proportional to its
count $R$, that is $R\times S= C$, for some universal constant $C$. In
part this observation achieved such fame for it's ubiquity and
simplicity. Zipf offered a simple formula representing the
interrelations between the words of a language, a possible
quantitative lead on the psychological processes underpinning a
language - his hypothesis being the `principle of least
effort''\cite{Zipf04}.

Regardless of the meaning behind Zipf's observation, implicit in his
analysis was the assumption that words form the natural atomization
for the clauses of a text. This assumption offers a very convenient
separation of a corpus, i.e. a counting by words assures that no
portion of a text is accounted for more than once. Here we seek
alternate and natural ways to separate a text.  In particular, we
develop distributions of phrases which arise from text separations.

To date the most extensively developed corpus of higher order lexical
in the computational linguistics community is that created by the
Google corporation \cite{Google01}. Despite its recent appearance as
data in various scholarly articles\cite{Garcia01,Piantadosi01} , the
perspective taken by Google and much of the $n$-grams community lacks
a thorough consideration of the active function of non-alphanumeric
symbols, ignoring punctuative boundaries and essentially considering
sequences embedded with punctuation to be potentially interpretable
phrases. In the Google corpus the entity `it's' is considered a phrase
of three words, with the apostrophe given the same status as a
word. This perspective is stuck, straddling the character-atomic and
word-atomic views simultaneously. Furthermore, as a consequence of the
sliding-window method of counting $n$-grams, the resulting
distribution cannot be integrated into a single distribution
meaningfully (each word forms a portion of many different $n$-grams).

Previous work has shown that a Zipfian word distribution can only
allow for non-Zipfian phrase distributions \cite{Egghe01}.
We, as
\cite{Ha01} have done, take a posteriori that the word distribution is
in fact non-Zipfian (since the relationship does break down).
Though
\cite{Ha01} tried to extend Zipf's work to $n$-grams, their attempts
utilized the sliding-window method of counting $n$-grams, immediately
destroying the statistical integrity of an integrated distribution.

As an alternative to these approaches this work takes punctuation
characters as rigid boundaries, and then bends the notion of counting
to be probabilistic.
Behind the punctuation-boundary assumption is the
observation that punctuation gives the interpreter an understanding of
where interpretable phrases start or end.
By `counting' words in a
probabilistic way, we construct a distribution of phrases respectful
to the distribution of words in a text.


Suppose now that of the (infinite?) collection of possible
interpretations of a clause, some reasonable interpretations are
modeled by the various separations, or, partitions of that clause.  In
particular, we assume that different partitions correlate to different
inflections, which correlate to different meanings.  We formalize this
as follows: given a clause $\sigma\in S$ of order $n$, define the
\emph{naive partition liklihood}, $p_\sigma(\tau)$, for any
\emph{full} (contiguous) phrase $\tau\preceq\sigma$, to be the
probability that a partition chosen at random (among the possible
$2^{n-1}$) isolates $\tau$.

As an example we let $\sigma=(let's\:eat\:grandma)$.  The possible
partitions of a clause nicely represented through the placement of
commas~\ref{fig:textpart.grandma}.
Of the four possible partitions of $\sigma$ we see
\begin{align*}
p_\sigma(let's)=1/2 \hspace{30pt}&\hspace{10pt} p_\sigma(let's\:eat)=1/4\\
p_\sigma(eat)=1/4 \hspace{36pt}&\hspace{10pt} p_\sigma(eat\:grandma)=1/4\\
p_\sigma(grandma)=1/2 \hspace{10pt}&\hspace{10pt} p_\sigma(let's\:eat\:grandma)=1/4.
\end{align*}
Among the first of things to note is, that each boundary a phrase
shares with the clause increases its partition probability by a factor
of 2. Also, since these probabilities are not mutually exclusive, they
do not sum to 1. In general we see that for any $\sigma\in S$ of order
$n$ and $\tau\preceq\sigma$, a full phrase of order $k$, that
$$
p_\sigma(\tau)
=
2^{b-(k+1)},
$$
where $b$ is the number of boundaries (left or right) that $\sigma$
and $\tau$ share (Appendix III).

\section{Local-context partitioning}

To rigorously develop this material we formally define a \emph{text}
to be a pair $(S,m)$, where $S$ is the space of finite-length
sequences of words from $W$, and $m:S\rightarrow \mathbb{Z}$ is the
clause multiplicity map, where a clause once again is defined to be
any sequence of words bounded by punctuation. Then, if we restrict our
analysis to any record of language, $m(\sigma)$ is the number of times
$\sigma\in S$ appeared as a clause in the record. With this, for any
$\tau\in S$ we define the \emph{naive expected count}, $c_\tau$, of
$\tau$ in the text to be
$$
c_\tau
=
\sum_{\sigma\in S}
m(\sigma)
p_{\sigma}(\tau),
$$
where $p_{\sigma}(\tau)=0$ if $\tau$ is not a full subsequence of
$\sigma$. Then, the \emph{expectation of occurrence} $E:2^S\rightarrow
\mathbb{Q}$ for any subset $A\subseteq S$ is well-defined by
$$
E(A)=\sum_{\tau\in A}c_\tau.
$$
The function $E$ is essentially describes the mass of a collection of
phrases in a text.

Given $\tau\in S$ with $\#\tau=n>0$, we now wish to determine which of
the $n$ possible local contexts $\tau$ represents. In section II we
define the first neighborhood of a phrase $\sigma$ to be the
collection of order $\#\sigma+1$ phrases constructible from
$\sigma$. Now, for a phrase $\tau$ the local contexts are precisely
the first neighborhoods that it belongs to. Therefore we take the
determination of a local context for $\tau$ to be the inference of
$\tau$'s construction. A reasonable approach, perhaps, assumes
construction of $\tau$ by $\sigma$ (i.e. assumes the context
$N_\sigma$) in proportion to $N_\sigma$'s expectation of
occurrence. Then, upon observing $\tau$ one assumes the context
$N_\sigma$ with probability
$$\frac{E(N_\sigma)}{\underset{\tau\in N_\eta}{\sum}E(N_\eta)}.$$
However, in the main body of this work, we take the simplification
that $\tau$ belongs to the context $N_\sigma$ with greatest
expectation of occurrence.

With our simplification, each phrase has a fixed context. So, we
consider the likelihood of a phrase's appearance given the presence of
a particular context. We define his quantity, the \emph{local
likelihood}, for any $\tau$ with context $N_\sigma$ to be
$$P(\tau\mid N_\sigma)=\frac{E(\tau)}{E(N_\sigma)}.$$
Local likelihood measurements form the basis of the partitioning
schemes in this work.

To develop a partitioning scheme we consider cognitive theories
\cite{Levy01}, which pose that the natural human syntactic parser
operates with limited resources. Typically the limited resource
assumed is memory. We may certainly average the local likelihood of
each clause partition and choose the one with greatest mean likelihood
(Appendix III). However, we must consider again that a clause of $n$
words has $2^{n-1}$ distinct partitions. Is it reasonable to expect a
human mechanism to scan all partitions? Given the memory-resource
limitation hypothesis, this brute force method seems unnatural, and
for long clauses totally unreasonable. Hence we expect that ideal
partitions of human language should be accessible through some
restriction of the search space.

To simplify the search problem we observe that English (along with
many other languages) relies heavily on a directional flow. This
property is exemplified by the `garden path' sentences of the English
language. Though these statements often lack punctuation and function
words, we find them instructive to understanding the limits of the
human lexical parser, most importantly the extent to which English
commonly reads from left to right. Consider the statement
\begin{center} 
  The prime number few.
\end{center} 
Since we immediately see the beginning of the phrase `the
prime number...'', we are forced to think of mathematics and attempt
to understand what sort of a number `few' is. However, a more
internally consistent interpretation considers `prime' to be a noun,
and verbs the word `number' into meaning `existence in a quantity
of''. Though this confusion is easily remedied through the insertion
of an implicit noun such as `people' into the statement `the prime
people number few'', the point is clear.  Given this nuance in the
English language we then expect that a reasonable mechanism `breaks
phrases off' from a clause, assessing the local likelihood of each
phrase from left to right. We refer to this process as \emph{serial
partitioning}~\ref{fig:textpart.grandma}.


\section{Experiment, Results, and Discussion}

Partitioning the Wikipedia.en, NY Times and Lyrics corpora via the
serial partitioning scheme with probabilistically determined contexts
(materials and methods discussed in Appendix II), we count the
occurrence of phrases up to order 10. Plotting the resulting phrase
distributions, we see that the Zipfian hypothesis is supported
remarkably well (Figure 1-A). The characteristic `knee' of the classic
word-separated distribution is greatly diminished and can be
attributed to the following. Words that appear frequently (see new,
york, and city in Figure 1-A) and are very likely to appear in
sequence will agglomerate into higher order units (see new york, york
city, and new york city in Figure 1-B). This process not only
diminishes the frequency of the individual words (suppressing the
knee), but introduces new, lower-


\noindent frequency atoms, which fill out the tail. Irrespective of
the phrase distribution's adherence to Zipf's relationship, we find
that the most frequently occurring phrases are interpretable, quite
representative of the source corpora and often idiomatic (Table 1).

Investigation to the Zipfian hypothesis immediately begs the question:
why a scaling exponent of $-1$? A well-known measure quite relevant to
this is that of Shannon and others \cite{Kolmogorov01, Shannon01},
known as information content.  This measure is most easily understood
as the average quantity of bits (when the logarithm is taken base 2)
per symbol required to efficiently store or represent a message.

The average representation-information required per symbol in a
Zipfian language of $N$ atoms with rank-decreasing frequencies
$\{p_k\}_{k=1}^N$, and scaling exponent $\gamma>1$ is given by the sum
$$-\sum_{k=1}^{N}p_k\log{p_k}=\frac{\gamma}{M_N(\gamma)}\left[\frac{\log{M_N(\gamma)}}{\gamma M_N(\gamma)}+\sum_{k=1}^{N}\frac{\log{k}}{k^\gamma}\right],$$
where $M_N(\gamma)$ is the $N$th generalized harmonic number of
exponent $\gamma$. Note that $M_N(\gamma)$ and the sum are finite in
the limit as $N\rightarrow\infty$ so long as $\gamma>1$. Hence a
growing language that maintains the Zipfian relationship may expand
its vocabulary indefinitely, i.e. the number of meaningful atoms may
increase without bound while the average representation-information
per symbol approaches a finite limit.  Furthermore, the value of the
theoretical, infinite sum, may become arbitrarily large as
$\gamma\rightarrow1$, meaning that such a language may maintain
discernible atoms which, on average, may represent any quantity of
information thinkable. For a system of encoding this seems to be a
powerful and provocative feature.
 
Regardless of what the information measure represents and what the
slope of a phrase distribution means, we once again note the value of
the context-dependent parser in application. Without any necessity for
parallel corpora, this development provides a novel MWE detection
technique.  From any serially encoded text, this method extracts the
likely lexical units, which in turn may be assessed for proper
translation.

\section{Sub-word partitioning}

As explained in [citation needed] a language posses multiple scales for representing meaning.
So far in this work, we have partitioned units at the word-sequence scale.
We can, however, look inside words (i.e. look at words like clauses), and partition letter-sequence units.
This partitioning can be done analagous to the method for phrases,
or, since word formation rules are more stringent
(e.g. pronouncable English units generally contain vowels and are composed of graphemes),
we can leverage known sub-word structure to obtain more realistic partitions.

Nevertheless,
the goal remains a partition that carries meaning.
At the word-seqence scale there is no qantative way to determine if a phrase is meaningful.
However, at the sub word scale we know nearly all combining forms [cite combining forms list],
in addition to roots, prefixes, and suffixes.
So, we present the sub-word partitioning results from two variants of the serial partitioning algorithm.
in addition to standard algorithm and random partitioning (see Table 1).
Each of these variations are performed by only taking expected counts from partitions adherent certain restrictions.
The first restriction requires that all partition components possess a vowel,
and the second requires that all partition componets possess a vowel or be a grapheme.

\begin{center}

{Coverage of Sub-Words Partitioned}\newline


{\fontsize{6.8}{8.16}\selectfont

\begin{tabular}{|c|c|c|c|c|}
\hline Source & Random & None  & Vowel & Grapheme\\\hline
NYTimes& 62.42\% (2435) & 72.79\% (100) &68.26\% (226) & 72.16\%(89) \\\hline
Wikipedia& 63.42\% (3191)& 0 & 0 &0 \\\hline
\end{tabular}
}
\end{center}

\section{Glossory of terms}
{\footnotesize
\textbf{Clause:} Any phrase in a corpus that abuts punctuation characters.\newline
\textbf{Corpus:} A sequential record of language.\newline
\textbf{First neighborhood:} For a phrase, The collection of all sequences constructible by it though a single insertion.\newline
\textbf{Full sub-phrase:} Of a phrase, any contiguous sub-phrase of terms.\newline
\textbf{Lexical unit/atom:} A minimal-length interpretable phrase. \newline
\textbf{Local likelihood:} The conditional likelihood of a phrase in context.\newline
\textbf{Maximal Sequential Intersection:} For two phrases, any maximal length sub-phrase common to the pair.\newline
\textbf{Naive expected count:} Sum of naive partition likelihood for a phrase across a text.\newline
\textbf{Naive partition likelihood:} The probability that a partition of a clause delineates a full sub-phrase of the clause.\newline
\textbf{Occurrence expectation:} The sum of naive expected count of the phrases in a collection.\newline
\textbf{Order of a phrase:} The number of terms in a phrase.\newline
\textbf{Partition-mean local likelihood:} The harmonic mean of a partition's components' local likelihood.\newline
\textbf{Partition of a clause:} A collection of full sub-phrases that when composed in sequence produce the clause.\newline
\textbf{Phrase:} Any sequence of words.\newline
%%\textbf{Positionally disjoint sub-phrases:} For a phrase, sub-phrases whose terms have differing indices.\newline
\textbf{Separated corpus:} A distribution of phrases that arises when every clause in a corpus is partitioned.\newline
\textbf{Sub-phrase:} For a phrase, any other phrase comparable in the partial order of no larger order.\newline
\textbf{Terms:} The (not possibly non-unique) words composing a phase.\newline
\textbf{Text:} Representing a corpus, $S$ paired with a map $m$ into the integers describing the multiplicity of occurrence of the clauses in the corpus. \newline
\textbf{Word:} Character sequences delineated by spaces and the null word, $\varnothing$.
}




\begin{thebibliography}{1}

\bibitem{Brame01} Brame, \emph{Recursive categorical syntax and morphology I: Semigroups, monoids and categories}, Linguistic Analysis,1984.

\bibitem{Egghe01} Egghe, \emph{On the law of Zipf-Mandelbrot for multi-word phrases}, Journal of the American Society for Information Science and Technology, 1999.

%%\bibitem{Frank01} Frank, Jaeger \emph{Speaking rationally: Uniform information density as an optimal strategy for language production}, Proceedings of the 30th Annual Meeting of the Cognitive Science Society, 2008.

\bibitem{Garcia01} Garcia, Garas and Schweitzer, \emph{Positive words carry less information than negative words}, 2012.

\bibitem{Google01} \emph{All Our N-gram are Belong to You},\\ http://googleresearch.blogspot.com/2006/08/all-our-n-gram-are-belong-to-you.html
\bibitem{Ha01}Ha, Hanna, Ming and Smith, \emph{Extending Zipf's law to n-grams for large corpora}, Artificial Intelligence Review, 2009.

\bibitem{Kolmogorov01} Kolmogorov, \emph{Three approaches to a quantitative definition of information}, Problems of Information Transmission, 1965.

\bibitem{Lakoff01}Lakoff and Johnson, \emph{Metaphors We Live By}, Univ. of Chicago Press, 1980.
\bibitem{Lakoff02}Lakoff, \emph{Women, Fire, and Dangerous Things}, University of Chicago Press, 1987.

\bibitem{Levy01} Levy, \emph{Probabilistic Models of Word Order and Syntactic Discontinuity}, Stanford
University Press, 2005.

%%\bibitem{Mandelbrot01} Mandelbrot, \emph{Information theory and psycholinguistics}, Scientific Psychology: Principles and Approaches, 1965.
%%\bibitem{Mandelbrot02} Mandelbrot, \emph{A note on a class of skew distribution function. analysis and critique of a paper by H.A. Simon}, Information and Control, 1955.
%%\bibitem{Mandelbrot03} Mandelbrot, \emph{Final note on a class of skew distribution functions: analysis and critique of a model due to H.A. Simon}, Information and Control, 1961.
%%\bibitem{Mandelbrot04} Mandelbrot, \emph{Post scriptum to `final note'}, Information and Control, 1961.

%%\bibitem{Miller01} Miller and Chomsky, \emph{Handbook of Mathematical Psychology II}, Wiley, pp. 419-491, 1963.

%%\bibitem{Nicolis} Nicolis and Tsuda, \emph{On the parallel between Zipf's Law and 1/f processes in chaotic systems possessing coexisting attractors: a possible mechanism for language formation in the cerebral cortex}, Progress of Theoretical Physics, 1989.

%%\bibitem{Orlov01} Orlov, \emph{Why, how, and when does the Zipf-Mandelbrot law fail?}, in Statistical Methods in Linguistics, 1977.

\bibitem{Piantadosi01} Piantadosi, Tily and Gibson, \emph{Word lengths are optimized for efficient communication}, PNAS, 2010
%%\bibitem{Piantadosi02} Piantadosi, Tily and Gibson, \emph{The communicative function of ambiguity in language}, Cognition, 2010.
%%\bibitem{Piantadosi03} Piantadosi, \emph{Learning and the language of thought}, PhD thesis, MIT, 2011.
%%\bibitem{Piantadosi04} Piantadosi, Tily and Gibson, \emph{Information content and word length in natural language}, (under review), 2012.

\bibitem{Reilly} Reilly and Kean \emph{Information content and word frequency in natural language: Word length matters}, Proc Natl Acad Sci USA, 2011.

%%\bibitem{Simon01} Simon, \emph{On a class of skew distribution functions}, Biometrika, 1955. 
%%\bibitem{Simon02} Simon, \emph{Some further notes on a class of skew distribution functions}, Information and Control, 1960.
%%\bibitem{Simon03} Simon, \emph{Reply to `final note' by Benoit Mandelbrot}, Information and Control, 1961.
%%\bibitem{Simon04} Simon, \emph{Reply to Dr. Mandelbrot's post scriptum}, Information and Control, 1961.

\bibitem{Sag01} Sag, Baldwin, Bond, Copestake, and Flickinger, \emph{Multiword Expressions: A Pain in the Neck for NLP}, Proc. of the 3rd International Conference on Intelligent Text Processing and Computational Linguistics.
\bibitem{Shannon01} Shannon, \emph{A Mathematical Theory of Communication}, Bell System Technical Journal, 1948.
%%\bibitem{Shannon02} Shannon, \emph{Prediction and entropy of printed English}, Bell System Technical Journal, 1951.

%%\bibitem{Yavuz} Yavuz, \emph{Zipf's law and entropy}, IEEE Transactions on Information Theory, 1974.

\bibitem{Zipf01} Zipf, \emph{Relative frequency as a determinant of phonetic change}, Harvard Studies in Classical Philology, 1929.
\bibitem{Zipf02} Zipf, \emph{the psycho-biologogy of language: an introduction to dynamic philology}, The M.I.T. Press, 1935.
%%\bibitem{Zipf03} Zipf, \emph{National Unity and Disunity: The Nation As a Bio-Social Organism}, Principia Press, 1941.
\bibitem{Zipf04} Zipf, \emph{Human Behavior and the Principle of Least Effort}, Addison-Wesley, 1949. 

%%\bibitem{Zornig01} Zornig and Altmann, \emph{The repeat rate of phoneme frequencies and the Zipf-Mandelbrot law}, Glottometrika, 1983.

\end{thebibliography}





We further argue that this view of phrases as the meaningful unit of
language reveals the English language possesses
optimal information content with 
the ability to describe an infinite set of things.  Meh.

minimal meaningful phrases rather than individual words,





that the frequency of words is inversely proportional
the remarkable 
to their rank



and its origins

Zipf's law is famous, much debate over its origin
-1 exponent
 
 
We show how Zipf's law applies to English for
meaningful phrases rather than words
 




The words of human languages 
usage frequency
exhibit  Zipf's law
The Zipf's law is 

Zipf's law is 

Here, we employ a simple probabilistic algorithm
that identifies meaningful phrases by context in large-scale
corpora.
We use our parsing scheme to extend the work
of Zipf  to distributions of English phrases.
We find Zipf's power-law hypothesis holds for phrases
rather than words, 
a strong argument for the semantic correlation of our technique.
We find strong support for our observations 
with a modified rich-gets-richer 
mechanism for language evolution.

In doing so we meet Zipf's power-law hypothesis better than
with English words alone, and use phrase considerations to develop a
stochastic model akin to that of H. A. Simon \cite{Simon01}.  
Using
this stochastic model we identify parameter regimes which produce the
characteristic "bent knee" Zipf distributions, a strong argument for
the semantic correlation of our theory.


Words are easy.

The description of a language is historically split into two parts:
grammar and the lexicon.
While the former consists of rules that
govern composition, the lexicon is the wealth of words and phrases
understood through context by interpreters.
Though grammar provides us
with a formal system of analysis, in its rigidity it often fails to
represent the underlying cognitive processes by which the language
itself likely forms.


Given modern perspectives from the cognitive
linguistics community \cite{Lakoff01,Lakoff02}, it is important to ask


how we may represent and identify the interpreted units of the
lexicon, phrases which cannot be sensibly broken apart 
(e.g, ``War on Terror'' and ``New York City'')


\begin{figure*}[tbp!]
  \centering
  \includegraphics[width=1\textwidth, angle=0]{generatedLanguage44_2.pdf}
  \caption{
    A - Generated language word distribution. Open circles and dashed
    line are measured and smoothed from the derived scaling (Appendix IV).
    B - Generated language phrase distribution.  C - Shannon phrase (red)
    and word (black) distribution entropies over of time.  D - Grey bars
    indicate the percent of all units counted that each order possesses,
    and red bars indicate the percent of all unique units each order
    possesses.
  }
  \label{fig:textpart.generatedlanguagezipf}
\end{figure*}


