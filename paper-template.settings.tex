\lefthyphenmin=3
\righthyphenmin=2



\usepackage{amssymb}
\usepackage{amsmath}

%% helpful for complicated papers
%% in finding out which figures, etc. are not being referenced

%% \usepackage{refcheck}
%% also helpful, highlighting the value of adding taxonomy to labels:
%% grep -A 1 refcheck paper-template-revtex4.log | grep -A 1 fig:


%% links
\usepackage{url}
%% url style
\urlstyle{same}

%% allow hyphenation for urls with long many-hyphenated strings
\PassOptionsToPackage{hyphens}{url}\usepackage{hyperref}
\makeatletter
\g@addto@macro{\UrlBreaks}{\UrlOrds}
\makeatother

\usepackage{colortbl}

\definecolor{goodblue}{RGB}{0, 91, 187}
\usepackage{hyperref}
\hypersetup{
  colorlinks=true,
  allcolors=goodblue,
  urlcolor=goodblue,
  citecolor=goodblue,
  pdfborder={0 0 0},
  breaklinks=true,
}

%% exciting emphasis action
%% https://tex.stackexchange.com/questions/218418/book-title-in-revtex-bibliography
%% plain ulem breaks bibliography:
%% fix:
\usepackage[normalem]{ulem}

%%%%%%%%%%%%%%%%%%%%%%%%%%%%%%%%%%%%%%%%%%%%%%%%%%%
%% color table with fix for revtex4
%%
%% see
%% https://tex.stackexchange.com/questions/102665/revtex4-1-and-colortbl-doesnt-fill-cells
%%%%%%%%%%%%%%%%%%%%%%%%%%%%%%%%%%%%%%%%%%%%%%%%%%%


\makeatletter

\def\CT@@do@color{%
  \global\let\CT@do@color\relax
  \@tempdima\wd\z@
  \advance\@tempdima\@tempdimb
  \advance\@tempdima\@tempdimc
  \advance\@tempdimb\tabcolsep
  \advance\@tempdimc\tabcolsep
  \advance\@tempdima2\tabcolsep
  \kern-\@tempdimb
  \leaders\vrule
  %^^A                     \@height\p@\@depth\p@
  \hskip\@tempdima\@plus  1fill
  \kern-\@tempdimc
  \hskip-\wd\z@ \@plus -1fill }
\makeatother

%%%%%%%%%%%%%%%%%%%%%%%%%%%%%%%%%%%%%%%%%%%%%%%%%%%

%% colors
\usepackage{xcolor}
\newcommand{\tc}[2]{\textcolor{#1}{#2}}
\newcommand{\tcn}[2]{\color[named]{#1}{#2}}
\definecolor{olivegreen}{rgb}{0.33333,.41961,0.18431}
\definecolor{forestgreen}{rgb}{0.13333,.5451,0.13333}
\definecolor{lightgrey}{rgb}{0.7,0.7,0.7}
\definecolor{verylightgrey}{rgb}{0.90,0.90,0.90}
\definecolor{veryverylightgrey}{rgb}{0.95,0.95,0.95}
\definecolor{grey}{rgb}{0.5,0.5,0.5}

% change to charcoal #36454F RGB={54,69,79}
% cinerous #98817B RGB={152,129,123}
% feldgrau #4D5D53 RGB={77,93,83}
% glaucous #6082B6 RGB={96,130,182}
% arsenic #3B444B RGB={59,68,75}

% header 51, 54, 126
\definecolor{headerblue}{HTML}{33367E}
\definecolor{unitednationsblue}{HTML}{4D88FF}

\definecolor{charcoal}{HTML}{36454F}
\definecolor{cinerous}{HTML}{98817B}
\definecolor{feldgrau}{HTML}{4D5D53}
\definecolor{glaucous}{HTML}{6082B6}
\definecolor{arsenic}{HTML}{3B444B}
\definecolor{xanadu}{HTML}{738678}

\definecolor{firebrick}{HTML}{B22222}
\definecolor{orangered}{HTML}{FF4500}
\definecolor{tomato}{HTML}{FF6347}

\definecolor{purpletaupe}{HTML}{3B444B}
% arsenic #3B444B RGB={59,68,75}
%% \definecolor{alertblue}{HTML}{3B444B}


%%%%%%%%%%%%%%%%%%%%%%%%%%%%%%%%%%%%%%%%%%%%%%%%%%%%%%%%%%%%%
%% build papers out of todo lists and first lines of paragraphs
%% (think Twitter threads or code documentation)
%%
%% also:
%% https://english.stackexchange.com/questions/479547/word-for-using-extra-line-breaks-to-improve-readability
%%
%% and use newlines as needed
%%%%%%%%%%%%%%%%%%%%%%%%%%%%%%%%%%%%%%%%%%%%%%%%%%%%%%%%%%%%%

\newcommand{\todo}[1]{\noindent\textcolor{goodblue}{{$\Box$ #1}}}
\newcommand{\paragraphnutshell}[1]{\noindent\textcolor{firebrick}{{Paragraph: #1}}}

%% paragraphs:

%% adjusted from https://tex.stackexchange.com/questions/469864/change-font-for-the-paragraph-command
%% normal or maybe bold text for first line
\usepackage{titlesec}
%% \titleformat*{\paragraph}{\mdseries}
\titleformat*{\paragraph}{\bfseries}

%% new lines:

%% https://english.stackexchange.com/questions/479547/word-for-using-extra-line-breaks-to-improve-readability
%% new sense-line
%% switch off for stupid journals
\newcommand{\newphrase}{\newline}

%%%%%%%%%%%%%%%%%%%%%%%%%%%%%%%%%%%%%%%%%%%%%%%%%%%%%%%%%%%%%
%% general stuff
%%%%%%%%%%%%%%%%%%%%%%%%%%%%%%%%%%%%%%%%%%%%%%%%%%%%%%%%%%%%%

\usepackage{graphicx,epsfig,verbatim,enumerate}

\usepackage{ifthen}

\usepackage{longtable}

\usepackage{mathtools}
%% \usepackage{abraces}

%% \usepackage{tikz-cd}
%% \usetikzlibrary{arrows}
%% \tikzset{
%%   commutative diagrams/.cd,
%%   arrow style=tikz,
%%  diagrams={>=space}}

\newboolean{twocolswitch}

\usepackage{array}

% ragged right in arrays
\newcommand{\PreserveBackslash}[1]{\let\temp=\\#1\let\\=\temp}
\let\PBS = \PreserveBackslash

%% blackboard font efforts; not good enough
%% would need to go before amssym
%% 
%% blackboard font for matrices
%% https://tex.stackexchange.com/questions/3473/blackboard-bold-variants-for-greek-letters
%% \usepackage[bbgreekl]{mathbbol}

%% "hollow" letters
%% https://tex.stackexchange.com/questions/360226/how-to-produce-hollow-greek-letters-in-math-mode


%%%%%%%%%%%%%%%%%%%%%%%%%%%%%%%%%%%%%%%%%%%%%%
%% notation (accumulated)
%%%%%%%%%%%%%%%%%%%%%%%%%%%%%%%%%%%%%%%%%%%%%%

\newcommand{\avg}[1]{\left\langle#1\right\rangle}
\newcommand{\tavg}[1]{\langle#1\rangle}

\newcommand{\LavgHa}[2]{\tavg{L_{#1,#2}}}
\newcommand{\LavgHaN}[2]{\tavg{L_{#1,#2}}}
\newcommand{\Lavg}{\tavg{L}}

\newcommand{\sindex}[1]{}
\newcommand{\nindex}[1]{}

\newcommand{\etal}{\textit{et al.}}
\newcommand{\www}[1]{\url{#1}}
\newcommand{\req}[1]{(\ref{#1})}
\newcommand{\Req}[1]{Eq.~(\ref{#1})}

% lettrines
\usepackage{lettrine}

\newcommand{\tbf}{\textbf}
\newcommand{\tit}{\textit}

%% differential equations and integrals
\newcommand{\dee}[1]{\textnormal{d}#1}
\newcommand{\pdiff}[2]{\frac{\partial #1}{\partial #2}}
\newcommand{\pdiffsq}[2]{\frac{\partial^{2} #1}{{\partial #2}^{2}}}
\newcommand{\diff}[2]{\frac{{\rm d}#1}{{\rm d}#2}}
\newcommand{\diffsq}[2]{\frac{{\rm d}^{2}#1}{{\rm d} {#2}^{2}}}
\newcommand{\tdiff}[2]{\textnormal{d} #1/\textnormal{d} #2}
\newcommand{\tdiffsq}[2]{\textnormal{d}^{2} #1/\textnormal{d} {#2}^{2}}
\newcommand{\tpdiff}[2]{\partial #1/\partial #2}
\newcommand{\tpdiffsq}[2]{\partial^{2} #1/\partial {#2}^{2}}
\newcommand{\postdee}[1]{\,\textnormal{d}#1}


\newcommand{\Prob}[1]{{\rm Pr}\left(#1\right)}

%%%%%%%%%%%%%%%%%%   
%% matrices
%%%%%%%%%%%%%%%%%%   

\newcommand{\colvec}[1]{
  \left[
    \begin{array}{c}
      #1
    \end{array}
  \right]
}

\newcommand{\rowop}[3]{
  {
    \scriptsize
    \begin{array}{c}
      \mbox{\Large $\leadsto$} \\
      \mbox{#1'} \\
      \mbox{= #1 -} \\ 
      \mbox{#2 #3} \\
    \end{array}
  }
}

\newcommand{\rowoptwobytwo}[3]{
  {
    \scriptsize
    \begin{array}{c}
      \mbox{\Large $\leadsto$} \\
      \mbox{#1'} 
      \mbox{= #1 -}
      \mbox{#2 #3} \\
    \end{array}
  }
}

\newcommand{\rowopeq}[3]{
  {
    \scriptsize
    \begin{array}{c}
      \mbox{\Large =} \\
      \left\lgroup
        \mbox{\tiny
          \begin{tabular}{c}
            #1' =  \\
            #1 - #2 #3 \\
          \end{tabular}
        }
      \right\rgroup
    \end{array}
  }
}

\newcommand{\rowswapeq}[2]{
  {
    \scriptsize
    \begin{array}{c}
      \mbox{\large =} \\
      \left\lgroup
        \mbox{\tiny
          #1 $\leftrightarrow$ #2
        }
      \right\rgroup
    \end{array}
  }
}

\newcommand{\rowdivthreebythree}[3]{
  {
    \scriptsize
    \begin{array}{c}
      \mbox{\Large $\leadsto$} \\
      \mbox{row 1'} = \mbox{row 1/(#1)}\\
      \mbox{row 2'} = \mbox{row 2/(#2)}\\
      \mbox{row 3'} = \mbox{row 3/(#3)}\\
    \end{array}
  }
}

\newcommand{\rowdiv}[2]{
  {
    \scriptsize
    \begin{array}{c}
      \mbox{\Large $\leadsto$} \\
      \mbox{#1'} = \mbox{#1/(#2)}\\
    \end{array}
  }
}

\newcommand{\rowdivtwobytwo}[2]{
  {
    \scriptsize
    \begin{array}{c}
      \mbox{\Large $\leadsto$} \\
      \mbox{row 1'} = \mbox{row 1/(#1)}\\
      \mbox{row 2'} = \mbox{row 2/(#2)}\\
    \end{array}
  }
}

\newcommand{\rowdivtwobytwoeq}[2]{
  {
    \scriptsize
    \begin{array}{c}
      \mbox{\Large =} \\
      \mbox{row 1'} = \mbox{row 1/(#1)}\\
      \mbox{row 2'} = \mbox{row 2/(#2)}\\
    \end{array}
  }
}

\newcommand{\rowdivthreebythreeeqR}[3]{
  {
    \scriptsize
    \begin{array}{c}
      \mbox{\Large =} \\
      \mbox{R1'} = \mbox{R1/(#1)}\\
      \mbox{R2'} = \mbox{R2/(#2)}\\
      \mbox{R3'} = \mbox{R3/(#3)}\\
    \end{array}
  }
}


\newcommand{\rowdivtwobytwoeqR}[2]{
  {
    \scriptsize
    \begin{array}{c}
      \mbox{\Large =} \\
      \mbox{R1'} = \mbox{R1/(#1)}\\
      \mbox{R2'} = \mbox{R2/(#2)}\\
    \end{array}
  }
}


\newcommand{\rowonediv}[1]{
  {
    \scriptsize
    \begin{array}{c}
      \mbox{\Large $\leadsto$} \\
      \mbox{row 1'} = \mbox{row 1/#1}\\
    \end{array}
  }
}

\newcommand{\mymatrix}[2]{
  \left[
    \begin{array}{#1}
      #2
    \end{array}
  \right]
}

%% \newcommand{\matrix}[1]{\mathbf{#1}}
%% \newcommand{\textmatrix}[1]{#1}
%% \renewcommand{\m}[1]{\mathbf{#1}}
%% \newcommand{\m}[1]{\mathbf{#1}}

\newcommand{\textmatrix}[1]{\mathbf{#1}}

\newcommand{\Axb}{\mathbf{A}\vec{x}=\vec{b}}
\newcommand{\Axzero}{\mathbf{A}\vec{x}=\vec{0}}
\newcommand{\A}{\mathbf{A}}
\newcommand{\AT}{\mathbf{A}^{\rm T}}
\newcommand{\CA}{C(\mathbf{A})}
\newcommand{\NA}{N(\mathbf{A})}
\newcommand{\CAT}{C(\mathbf{A}^{\rm T})}
\newcommand{\NAT}{N(\mathbf{A}^{\rm T})}

\newcommand{\I}{\mathbf{I}}
\newcommand{\R}{\mathbf{R}}

\newcommand{\augAb}{[\A \, | \, \vec{b} \, ]}

\newcommand{\rref}[1]{\R_{#1}}

\newcommand{\lmat}{\left[
    \begin{array}
    }
    \newcommand{\rmat}{\end{array}
  \right]
}

%%%%%%%%%%%%%%%%
%% contagion
%%%%%%%%%%%%%%%%

% \newcommand{\kstar}{d^{\ast}}
% \newcommand{\kstari}{k_i^\ast}
\newcommand{\kstar}{d^\ast}
\newcommand{\kstari}{d_i^\ast}

\newcommand{\dstar}{d^\ast}
\newcommand{\dstari}{d_i^\ast}

\newcommand{\phifix}{{\phi^{\ast}}}
\newcommand{\phifixc}{{\phi_c^{\ast}}}
\newcommand{\phifixb}{{\phi_b^{\ast}}}

\newcommand{\phiiactive}{\phi_{i,t}}

\newcommand{\phiiup}{{\phi_{i,\rm on}}}
\newcommand{\phiidown}{{\phi_{i,\rm off}}}

\newcommand{\phiup}{{\phi_{\rm on}}}
\newcommand{\phidown}{{\phi_{\rm off}}}


\newcommand{\Gfun}{G}

\newcommand{\dstardist}{g}
\newcommand{\dosedist}{f}
\newcommand{\dosedistk}{f^{k\star}}

% chaotic contagion
\newcommand{\edgeinfprob}{\rho}
\newcommand{\nodeinfprob}{\phi}

\newcommand{\avgdegree}{k_{\rm avg}}

\newcommand{\stateA}{S_{0}}
\newcommand{\stateB}{S_{1}}

\newcommand{\lcce}{H}

%% \newcommand{\effectivecount}{f_{\textnormal{eff}}}

\newcommand{\effectivecount}{f_{q,\textnormal{exp}}}
\newcommand{\effectivecountq}[1]{f_{#1,\textnormal{exp}}}

\newcommand{\simonrho}{\rho}
\newcommand{\zipfexponent}{\alpha}

\newcommand{\partitionprob}{q}
\newcommand{\partitiontype}[1]{$\partitionprob$$=$$#1$}

\newcommand{\onehalf}{\frac{1}{2}}
\newcommand{\onequarter}{\frac{1}{4}}

\newcommand{\clgamma}{\hat{\zipfexponent}}
\newcommand{\xmin}{r_{\textnormal{max}}}
\newcommand{\kstat}{D}
\newcommand{\clp}{\textnormal{$p$-value}}
\newcommand{\sigamma}{1-\simonrho}


